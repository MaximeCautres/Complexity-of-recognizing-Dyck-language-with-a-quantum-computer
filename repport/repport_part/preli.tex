In order to understand from where comes the $\tilde{\Theta}(\sqrt{n})$ quantum
query complexity, it is mandatory to start by understanding the
Trichotomy theorem \cite{trichotomy_not_andris} and its dependency on Regular
languages and star free languages.
After that, to understand how to find a lower bounds, it is necessary to explore
the different adversary methods \cite{adversary_equivalence}.

\subsection{Quantum query for regular languages.}

In the article \cite{trichotomy_not_andris}, Aaronson, Grier and Schaeffer presented
a really interesting algebraic characterization of regular languages based on the
recognition by monoids and the syntactic congruence. This totally new to me definition
give me some hard time in order to understand it correctly.

\subsubsection{Regular languages}

Usually, the set of regular languages $\mathcal{R}$ on the alphabet $\Sigma$ is defined as the smallest
fixed point of the function F (the function that computes concatenations, unions, and Kleene stars)
including $\{\varnothing\} \cup \{\varepsilon\} \cup(\cup_{l\in\Sigma}\{l\})$
with F equal
\begin{align*}\label{eq:F}
    F(X) = & \{AB\  \forall (A,B)\in X^2\}                                                   \\
           & \cup \{A\cup B\ \forall (A,B) \in X^2 \}                                        \\
           & \cup \{A\cap B\ \forall (A,B) \in X^2 \}\ \texttt{\color{gray!50}$\#$ optional} \\
           & \cup \{A^*\ \forall A \in X \}.
\end{align*}

But here, the more convenient way to characterize regular language is with their
algebraic characterization. Indeed, a regular language is always a pre image of a subset
of a monoid under monoid homomorphism. A monoid is a 3-tuple of a set M, an
internal associative binary operation and finally the identity element associated to the
operation. A monoid homomorphism is a map from a monoid to another that preserve the
operation and the identity. Usually, to get the monoid representation of a regular
language, the first step is to compute the syntactical monoid. The syntactical monoid is
obtained by dividing $\Sigma^*$ by the following equivalence relation called syntactic
congruence
\[x \sim_L y \Leftrightarrow \forall (u, v) \in \left(\Sigma^*\right)^2, (uxv \in L \Leftrightarrow uyv \in L).\]
This equivalence relation is a congruence relation as the equivalence class can be multiplied
(i.e. if $x\sim_L y$ and $u \sim_L v$ then $xu \sim_L yv$). Now, it is necessary to get the monoid
homomorphism. For that it is sufficient to take the homomorphism that map an element to its
congruence class. Inside a congruence class, none or every element of the class is in $L$.
Indeed,  if $x$ is in $L$ and $x \sim_L y$ then for all
$(u, v)$ in $\left(\Sigma^*\right)^2, (uxv$ in $L \Leftrightarrow uyv$ in $L)$ thus
$x$ in $L$ implies $y$ in $L$ and finally as $x$ is in $L$ then $y$ is also in $L$.
Now, it is sufficient to prove that the syntactical monoid is of finite size.
{\color{red} ma be put the proof if time else say folklore.}

Finally, every regular language can be recognized by finite monoid.

\subsubsection{Star free languages}\label{ssec:starfree}

The set of star free languages is a really well studied subset of the regular languages.
Its definition differs a little from regular languages' one as the Kleene star is replace
by the complement operation (note $\overline{L}$). So star free languages are defined
as the smallest fixed point of the function F' (the function that computes concatenations, unions
and complements) and such that it includes
$\{\varnothing\} \cup \{\varepsilon\} \cup(\cup_{l\in\Sigma}\{l\})$.
This restriction does not imply that every star free language is finite. Indeed,
$\Sigma^*$ can be written $\overline{\varnothing}$. For example the language on
$\Sigma = \{1, 2, 3\}$ described with the regular expression $\Sigma^*20^*2\Sigma^*$ can be written
as following in the star free way
$\overline{\varnothing}2\overline{\overline{\varnothing}\Sigma{\setminus}\{0\}\overline{\varnothing}}2\overline{\varnothing}$.

As for regular languages, it exists an algebraic characterization for star free
languages. Let $M$ be a monoid, $M$ is said to be aperiodic if for every $x$ in
$m$ it exits a positive integer $n$ such that $x^n=x^n+1$. A theorem proved by
Schützenberger \cite{Schtzenberger1965OnFM} state that a language is recognized
by a finite aperiodic monoid if and only if if is star free.

Good examples of star free languages are the Dyck word languages with bounded heights.
It is first easy to have a finite automata that recognize Dyck word of height $k$ by
putting one state for each integer from 0 up to $k$. However, the belonging to the star
free regular languages is more delicate to prove. It has been done done by Italian
researcher in \cite[1978]{dyck_height_bound_star_free}.

\subsubsection{Trichotomy theorem}

\begin{theorem}[Aaronson, Grier and Schaeffer \cite{trichotomy_not_andris}]
    Every regular language has quantum query complexity
    \(0,\Theta(1), \tilde{\Theta}(\sqrt{n}),\) or \(\Theta(n) \)
    according to the smallest class in the following hierarchy that contains
    the language.
    \begin{itemize}
        \item Degenerate: One of the four languages \(\varnothing, \varepsilon, \Sigma^*\), or \(\Sigma^+\).
        \item Trivial: The set of languages which have trivial
              \footnote{A language $L$ is said to be trivial if and only if it exists 2 finite size alphabets
                  $L_1$ and $L_2$ such that $L = L_1 \Sigma^* L_2$.}
              regular expressions.
        \item Star free: The set of languages which have star-free regular expressions.
        \item Regular: The set of languages which have regular expressions.
    \end{itemize}
\end{theorem}

This theorem is really important as it give a good idea on the quantum query complexity
of many language recognitions. Moreover, the classes are now clearly defined so it is now
easier to know where is a problem compared to the classes in the introduction. However,
it does not give the exact quantum query complexity because for star free languages
the result is given using \(\tilde{\Theta}\). The $\tilde{\Theta}(\sqrt{n})$ means that
the quantum query complexity of any star free regular language is in
$\Theta(\sqrt{n}(\log_2(n))^p)$ for some $p$ a non negative constant.
As it is known that \Dyck{k} languages are star free, it is an interesting problem
to find the power of $\log_2(n)$ depending on the value of $k$.

\subsection{The bounds for \Dyck{k,n} problem}

The trichotomy theorem state that \Dyck{k,n} language has a quantum query complexity
in $\tilde{\Theta}(\sqrt{n})$. The $\Theta$ means that the best possible algorithm
is both a $\tilde{O}(\sqrt{n})$ and a $\tilde{\Omega}(\sqrt{n})$. So, a common method
to find the power of $\log_2(n)$ depending on $k$ is the squeeze theorem. More
precisely, the quantum query complexity of \Dyck{k, n} is trapped between a lower
and an upper bound for quantum query complexity. So it is possible to deduce
the quantum query complexity from an increasing sequence of lower bound and
a decreasing sequence of upper bound such that they share the same limit $l$ with $l$
equals to $Q(\Dyck{k,n})$. How to find this sequence ?
\begin{itemize}
    \item \textbf{For the lower bound sequence.} Some of the most important tools to compute
          lower bounds are called \textbf{quantum adversary methods} and have been invented by Ambainis
          \cite{andris_adversary_methode}. This tools can compute
          lower bounds more or less tight and some of this adversary methods have really
          useful property such as being compatible with a certain type of composition
          of problems. This lead to the \textbf{reduction methods} that can compute
          a lower bounds from different but easier problem's ones.
    \item \textbf{For the upper bound sequence.} The main method is to \textbf{find quantum
              query algorithms} that are more and more efficient. An other way is also
          \textbf{by reduction} to a more difficult problem with a
          good enough upper bound.
    \item \textbf{For the same limit.}
          The idea is to do an iterative process where each step improves successively
          each bound until only one will continue to be improved. Finally, both
          bounds may end up matching. However, before my internship, both \Dyck{k,n}'s
          lower and upper bounds where stuck to $\Omega\left(c^k\sqrt{n}\right)$ for some
          constant $c$ greater than 1 and $O\left(\sqrt{n}\left(\log_2(n)\right)^{0.5k}\right)$.
\end{itemize}


\subsubsection{Lower bound on the quantum query complexity of \Dyck{k, n}}

\paragraph*{\textbf{The quantum adversary methods:}}
This part as for goal to present some quantum adversary methods and their usages.

\subparagraph*{\textbf{The adversary method.}} The adversary method is a classical
way to prove lower bound in classical algorithmic. Indeed, the idea behind the classical
adversary is to prove that for every algorithm, it exist an entry such that the algorithm
cannot decide in less unit of complexity than the lower bounds. In general, during
the algorithm execution, the adversary modifies entry values that have not been already used
in order to increase the execution time. This classical adversary work great for classical
algorithm but is not really useable on  quantum algorithms.

\subparagraph*{\textbf{The super basic quantum adversary method.}}
In order to recognize a language it is mandatory to be able to distinguish between
a valid word $v$ and an invalid word $w$. So at the output of the quantum query
algorithm, the states $\ket{\psi^T_v}$ and $\ket{\psi^T_w}$ should be distinguishable thus
$\vert \langle \psi_v^T \vert \psi_w^T \rangle \vert < \frac{2}{3}$. How do they
will differs? The quantum query algorithm start with
$\ket{\psi^0_v} = \ket{\psi^0_w} = \ket{\psi_{start}}$. Moreover, the inner product
of both states isn't affected by $U_i$ gates because there are unitary. However, the $Q_i$
gates affect this inner product as the $Q_i$'s behaviors are depending on the input.
Let define the progress measure $\mathcal{P}$ such that
$\mathcal{P}(t) := \langle \psi_v^t \vert \psi_w^t \rangle$ where $\ket{\psi^t}$ represents
the state after $Q_t$. Let suppose it exists $d$ such that for all $0\leq t \leq T-1$,
$\vert \mathcal{P}(t+1)-\mathcal{P}(t)\vert \leq d$. It implies the following inequality
\[
    \mathcal{P}(T) = \underbrace{\mathcal{P}(0)}_{=1} + \sum_{i=0}^{T-1}\underbrace{\mathcal{P}(i+1)-\mathcal{P}(i)}_{\geq -d}  \geq \mathcal{P}(0) - T\times d.
\]
Moreover, $\mathcal{P}(T)$ should be lower than $\frac{2}{3}$ so it implies that
$1 - T\times d$ should also be lower than $\frac{2}{3}$ and finally that
$T$ should be a $O(\frac{1}{d})$. This give a general idea of how to determined
a lower bound but it isn't precise enough to get a theorem. In its courses,
Ryan O'Donnell \cite{Odonnel_course} explained a simple to understand adversary method called the super
basic adversary method. This adversary allow to compute a lower bound
by using the following method:

\begin{theorem}
    Let define $YES$ the set equal to $f^{-1}(accepted)$ and $NO$ the set
    equal to

    $f^{-1}(rejected)$. If it exist two subset $Y\subseteq YES$ and
    $Z \subseteq NO$ such that:
    \begin{itemize}
        \item For each $y$ in $Y$, there are at least $m$ strings $z$ in $Z$ with dist$(y,z)=1$.
        \item For each $z$ in $Z$, there are at least $m'$ strings $y$ in $Y$ with dist$(y,z)=1$.
    \end{itemize}
    Then the quantum query complexity $Q(f)$ is in  $\Omega(\sqrt{m m'})$.
\end{theorem}

The proof of the theorem can be found in \autoref{proof:advmethod}. More generally,
the method of an adversary define a process to find the lower $d$ possible. The
super-basic adversary method as its name let suggest is simple but does not give
a tight lower bounds. For this, the method should be more complicated which results
in longer days on the board to find the parameters of the methods.

Let f be the function whose quantum query complexity is unknown. The two adversary methods
are described as following:
\begin{itemize}
    \item \textbf{The Basic adversary method by Ambainis \cite{andris_adversary_methode}}
          \begin{theorem}
              Let define $YES$ the set equal to $f^{-1}(accepted)$ and $NO$ the set
              equal to

              $f^{-1}(rejected)$. Moreover, let $Y\subseteq YES$, $Z \subseteq NO$, and
              let $\mathcal{R} \subseteq Y\times Z$ be a set of "hard to distinguish" pairs, such that:
              \begin{itemize}
                  \item For each $y$ in $Y$, there are at least $m$ strings $z$ in $Z$ with $(y,z)$ in $\mathcal{R}$.
                  \item For each $z$ in $Z$, there are at least $m'$ strings $y$ in $Y$ with $(y,z)$ in $\mathcal{R}$.
              \end{itemize}
              Also, let define $\mathcal{R}_i$ as a restriction from $\mathcal{R}$ such
              that $(y, z)$ is in $\mathcal{R}_i$ if and only if $(y, z)$ is in $\mathcal{R}$
              and $y_i \neq z_i$.
              In addition,
              \begin{itemize}
                  \item For each $y$ in $Y$ and $i$, there are at most $l$ strings $z$ in $Z$ with $(y,z)$ in $\mathcal{R}_i$
                  \item For each $z$ in $Z$ and $i$, there are at most $l'$ strings $y$ in $Y$ with $(y,z)$ in $\mathcal{R}_i$
              \end{itemize}
              Then the quantum query complexity $Q(f)$ is in  $\Omega(\sqrt{\frac{m m'}{ll'}})$.
          \end{theorem}
    \item \textbf{The general adversary method by \cite{}}
\end{itemize}

\paragraph*{\textbf{The reduction method:}}
\begin{itemize}
    \item how does it work
    \item how does it translate to quantum query complexity
    \item the result for a lower bound on \Dyck{k,n}
\end{itemize}

\subsubsection{Best known algorithm to recognize \Dyck{k, n}}

\paragraph*{\textbf{How to reject a word from \Dyck{k,n}}}
\begin{itemize}
    \item recall for the height $height$
    \item $\pm$ $k+1$ strings, minimal ones
\end{itemize}

\paragraph*{\textbf{explaination why it is hard for $k > 1$}}
\begin{itemize}
    \item There is an infinit number of rejecting strings to search for
\end{itemize}

\paragraph*{\textbf{Andris ambainis team solution}}
\begin{itemize}
    \item Give an detailed explaination of the algorithm that are in the annexe
    \item explain why every part of the algorithm is important.
          \begin{itemize}
              \item the search on $d$
              \item the log search to have limited error.
              \item \FALM{k}
          \end{itemize}
    \item
\end{itemize}