\subsection{A try to expand the new
    algorithm's first version to every $k$.}

The second version of the algorithm is a lot more simple than the previous one,
but it doesn't make the first version of the algorithm obsolete. Indeed, the
first algorithm was think with the goal to be easily modified to work with higher
value of $k$. Unfortunately, the key property that make the algorithm works
(i.e. after each letter with an even index, the height is equal to 0 or 2, both
being the bounds for the height) no more hold. So I didn't succeed to find a way
to generalized to greater value of $k$.

\subsection{A try for a new algorithm for any k.}
\label{subsec:dicho}

In order to reduce the quantum query complexity of \Dyck{k}, one idea was
to reduce the number of recursive calls during the execution. For that,
two different possibilities seemed possible, both were using the same
main principle: The recursive calls could be associated to nested loops,
whose variables would correspond to the choice of $d$ for each recursion level.
Indeed, in every recursion level, \FA{k} is called with $l$ and $r$ such that
$r-l\leq d_{k+1}$ with $d_{k+1}$ the value give to $d$ in the call of \FA{k+1}.
It is in reality more complicated but the goal is just to check if,
first it is possible to compute the quantum query complexity
of this model, and after to translate the method to get a better algorithm to
recognize \Dyck{k}.

\paragraph*{A natural graph structure on this (k-2)-tuples.} The initial
cases of the recursive algorithm  are for $k$ equal to 2, 3, so there
is only $k-2$ nested loops, thus $d_i$
associated variable, one for each recursive level. This $k-2$ variables can
be consider together as $k-2$-tuples. In every tuple, the following inequality
is verified $ d_{k+1} \geq d_k \geq \ldots \geq d_4$. Each tuple is a vertex
in a graph $(V,E)$. Edge $E$ are defined as following.
\[ (t_u,t_v) \in E \Leftrightarrow \exists ! i \in \{1, k-2\}, \left( (t_u)_i
    \neq (t_v)_i \land (t_v)_i = 2(t_u)_i\right)\]
A vertex of the graph is said to me marked if and only if
\begin{center}
    \begin{minipage}{.8\textwidth}
        it exists a $\pm (k+1)$-strings
        of length at most $(t_u)_1$ composed of 2 $\pm k$-strings of size at most
        $(t_u)_2$ composed them selves of 2 $\pm (k-1)$-strings of size at most
        $(t_u)_3$, ... and composed themselves of  2 $\pm 3$-strings of size at most
        $(t_u)_k-2$.
    \end{minipage}
\end{center}

This graph has an interesting property $\mathcal{P}$ for every vertex $t_u$.
If $t_u$ is marked then every descendent $t_v$ of $t_u$ in the graph is also
marked.
Without using this property, the $k-2$ nested loops are equivalent to a DFS that
search for a marked vertex. Then the quantum query complexity of this algorithm
would be the one of the quantum DFS multiplied by the one to check a marked vertex.
This give us a quantum query complexity in
$O\left(\sqrt{\log_2(n)^{k-2}}\times Q(\textrm{mark\ checking})\right)$.
The quantum query complexity of checking if a vertex is marked depends a lot
on what is allowed to do and the final quantum query complexity required.
For the nested loops, the goal is to have an algorithm that improve the quantum
query complexity of $O\left(\sqrt{n}\log_2(n)^{0.5(k-2)}\right)$. Thus, the quantum
query complexity of checking a vertex should be at most $O(\sqrt{n})$.
What does it mean to check a vertex? This is difficult to define as the current
model of nested loop is quite far from the original algorithm. But, one can do
some hypothesis and state that as for the original algorithm, it is only possible
to check easily $O(\sqrt{n})$ a second type of marks define as:
\begin{center}
    \begin{minipage}{.8\textwidth}
        it exists a $\pm (k+1)$-strings of length between $(t_u)_1/2$ and  $(t_u)_1$ composed of
        2 $\pm k$-strings of size between $(t_u)_2/2$ and $(t_u)_2$ composed them selves of 2
        $\pm (k-1)$-strings of size between $(t_u)_3/2$ and $(t_u)_3$, ... and composed themselves
        of  2 $\pm 3$-strings of size between $(t_u)_{k-2}$ and $(t_u)_{k-2}$.
    \end{minipage}
\end{center}

With this statement, it is not possible to use an algorithm whose probability
to check a node can be zero else as for the original one, it is not possible
to be sure of the answer. The only solutions are graph traversal that have
a continuous border between explored and non explored vertices (DFS or BFS
for example). So, in order to go around this constraint, it may be possible
to use slower algorithm to check the first type of mark if it implies a
non negligible improvement on the quantum query complexity of the exploration
as it is not really clear what would be the quantum query complexity of checking
a mark.

In order to present the the main way to reduce the cost of the traversal,
it is useful to order the vertex of the graph onto a discrete finite space
of dimension $k-2$ were every axis has a logarithmic scale (base 2) starting
from 1 to $2^{\lfloor \log_2(n) \rfloor}$.

In this space, a vertex $t_u$ has its coordinate on axis $i$ equal to $(t_u)_i$.
The goal here is to do a binary search on each
dimension of the $k-2$-tuple, in order to find a marked vertex.
In fact, the property $\mathcal{P}$ implies that if a vertex $t_u$
is marked then every vertex $t_v$ is marked if $t_u$ and $t_v$ differ
only on the j-th dimension and $(t_u)_j\leq (t_v)_j$, $t_v$. So on every
of the $k-2$ dimensions, it is possible to do a binary search because
it exists a value, depending on the value of all the first dimension such that
no vertex is marked before and all vertex are marked after. Thus, the quantum
query complexity of this multidimensional binary search would be
\[O\left(\sqrt{\log_2(\log_2(n))}^{k-2} \times Q(\textrm{marking\ checking})\right).\]

The improvement is quite spectacular if it is possible to find an algorithm
that check the mark of a vertex with a quantum query complexity that does not
counter balance. With more time, it would have been possible to define more
precisely what it means to check a mark and may be an algorithm could have been
found. It may also be possible to prove that such an algorithm cannot exits.

\subsection{A new adversary plus minus for \Dyck{k,n}}

The idea was first to search for a basic adversary (\autoref{th:basicadv}) for small value of $k$.
The efforts were focus on $k=2$. The goal was to find two sets of word from
$\{0,1\}^n$ that satisfy the requirement of the basic adversary. Moreover,
this two sets should give a lower bound of the form
$O(\sqrt{n}f(n))$ (for some function $f(n)$) in order to increase the current
lower bound of $O(\sqrt{n}c^2)$ (for some constant $c$).
This basic adversary returns lower bounds of the form
$\sqrt{\frac{mm'}{ll'}}$ where $m,m',l,$ and $l'$ (depending on $n$)
specify properties on both sets. I do not have found such sets,
more precisely, it is already difficult to think about a way
to define two sets such that their sizes increase nicely according
to a function in $n$. So, having to deal in the same time with
their associated values $m,m',l,$ and $l'$ made the task really difficult.


\subsection{A new reduction from easier problems}

In order to find a new lower bounds, one idea was to try new reduction.
For $k=2$, some languages have been tested but an optimal algorithm
have been found. For $k=3$, one of the most interesting language is defined
by $c^*(ac^*bc^*)^*$.
In order to reject a word $w$, it is sufficient to search if a substring $v$
from $bwa$ can be part of the language $ac^*a+bc^*b$ so this language is also
a star free language (the same arguments than the one for \Dyck{k} would work here).
This means that its quantum query complexity is in $\tilde{\Theta}(\sqrt{n})$.
Moreover, it is possible to reduce this problem to \Dyck{3} simply by doing
the rewritings $a \mapsto 00, b\mapsto 11$ and $c \mapsto 01$. So, finding the
quantum query complexity of this problem can be useful in order to get a lower
bound to the quantum query complexity of \Dyck{3}.