\subsection{An attempt to expand the new
    algorithm's first version to every $k$}

The second version of the fast algorithm is a lot simpler than the
previous one, but it doesn't make the first version of the algorithm
obsolete. Indeed, the first algorithm was thought with the idea of being
easily modified in order to work with higher value of $k$. Unfortunately,
the key property that make the algorithm works (i.e. after each letter with
an even index, the height is equal to 0 or 2, both being the bounds for the
height) no more holds. So I didn't succeed to generalized to greater value
of $k$ as the behaviors of intermediate height become too complex.

\subsection{An attempt for a new algorithm for any k}
\label{subsec:dicho}

To reduce the quantum query complexity of \Dyck{k}, one idea was
to reduce the number of recursive calls during the execution. For that,
two different possibilities seemed possible, both using the same
main principle: The recursive calls could be associated to nested loops,
whose variables would correspond to the choice of $d$ for each recursion level.
Indeed, in every recursion level, \FA{k} is called by \FA{k+1} with $l$ and $r$
such that $r-l\leq d_{k+1}$ with $d_{k+1}$ the value give to $d$ in the call
from \FA{k+1}. It is in reality more complicated but the goal is just to
check if, first it is possible to compute the quantum query complexity
of this model, and after to translate the method to get a better algorithm to
recognize \Dyck{k}.

\paragraph*{A natural graph structure on the (k-2)-tuples.} The initial
cases of the recursive algorithm are for $k$ equal to 2, 3, so there
are only $k-2$ nested loops, with one $d_i$ for each recursive level.
This $k-2$ variables can be consider together as $k-2$-tuples. In every
tuple, the inequality $ d_{k+1} \geq d_k \geq \ldots \geq d_4$ is verified
and each tuple is a vertex in a graph $(V,E)$. The edges $E$ are defined as
following.
\[ (t_u,t_v) \in E \Leftrightarrow \exists ! i \in \{1, k-2\}, \left( (t_u)_i
    \neq (t_v)_i \land (t_v)_i = 2(t_u)_i\right)\]
A vertex of the graph is said to me marked if and only if
\begin{center}
    \begin{minipage}{.8\textwidth}
        it exists a $\pm (k+1)$-strings
        of length at most $(t_u)_1$ composed of 2 $\pm k$-strings of size at most
        $(t_u)_2$ composed them selves of 2 $\pm (k-1)$-strings of size at most
        $(t_u)_3$, ... and composed themselves of  2 $\pm 3$-strings of size at most
        $(t_u)_k-2$.
    \end{minipage}
\end{center}

This graph has an interesting property $\mathcal{P}$ for every vertex $t_u$.
If $t_u$ is marked then every descendent $t_v$ of $t_u$ in the graph is also
marked.
Without using this property, the use of this $k-2$ nested loops is equivalent
to a DFS that search for a marked vertex. Then the quantum query complexity
of this algorithm would be the one of the quantum DFS multiplied by the one
to check a marked vertex. It gives us a quantum query complexity in
$O\left(\sqrt{\log_2(n)^{k-2}}\times Q(\textrm{mark\ checking})\right)$.
The quantum query complexity of checking if a vertex is marked depends a lot
on what is allowed to do and the final quantum query complexity required.
For the nested loops, the goal is to have an algorithm that improves the quantum
query complexity of $O\left(\sqrt{n}\log_2(n)^{0.5(k-2)}\right)$. Thus, the
quantum query complexity of checking a vertex should be at most $O(\sqrt{n})$.
But what does it mean to check a vertex? This is difficult to define as
the current model of nested loop is quite far from the original algorithm
and not really finish to define. But, one can do some hypothesis and state
that as for the original algorithm, it is only possible to check easily
($O(\sqrt{n})$) a second type of marks (which is not compatible with
the property $\mathcal{P}$) define as
\begin{center}
    \begin{minipage}{.8\textwidth}
        it exists a $\pm (k+1)$-strings of length between $(t_u)_1/2$ and  $(t_u)_1$ composed of
        2 $\pm k$-strings of size between $(t_u)_2/2$ and $(t_u)_2$ composed them selves of 2
        $\pm (k-1)$-strings of size between $(t_u)_3/2$ and $(t_u)_3$, ... and composed themselves
        of  2 $\pm 3$-strings of size between $(t_u)_{k-2}$ and $(t_u)_{k-2}$.
    \end{minipage}
\end{center}

With this statement, it is not possible to use an algorithm whose probability
to check a node can be zero otherwise, as for the algorithm by Andris Ambainis's
team, it is not possible to be sure enough of the answer. The only solutions
are graph traversal that have a continuous border between explored and
non explored vertices (DFS or BFS for example) and that becomes
faster using Grover. So, in order to go around this constraint, it may be
possible to use a slower algorithm to check the first type of mark if it
implies a non negligible improvement on the quantum query complexity of
the traversal as it is not really clear what would be the quantum
query complexity of checking the first type of marks (not well defined).

In order to present the main way to reduce the cost of the traversal,
it is useful to order the vertex of the graph onto a discrete finite space
of dimension $k-2$ were every axis has a logarithmic scale (base 2) starting
from 1 to $2^{\lfloor \log_2(n) \rfloor}$.

In this space, a vertex $t_u$ has its coordinate on axis $i$ equal to $(t_u)_i$.
The goal here is to do a binary search on each dimension of the $k-2$-tuple
in order to find a marked vertex. In fact, the property $\mathcal{P}$ implies
that if a vertex $t_u$ is marked then every vertex $t_v$ is marked if $t_u$ and
$t_v$ differ only on the j-th dimension and $(t_u)_j\leq (t_v)_j$.
So on every of the $k-2$ dimensions, it is possible to do a binary search
because it exists a value, depending on the values of all the first dimensions
such that no vertex is marked before and all vertices are marked after.
Thus, the quantum query complexity of this multidimensional binary search
would be
\[O\left(\sqrt{\log_2(\log_2(n))}^{k-2} \times Q(\textrm{marking\ checking})\right).\]

The improvement is quite spectacular if it is possible to find an algorithm
that check the mark of a vertex with a quantum query complexity that does not
counter balance the benefit. With more time, it would have been possible to
define more precisely what it means to check a mark and may be an algorithm
could have been found. It may also be possible to prove that such an algorithm
cannot exits.

\subsection{A attempt to find a new adversary plus minus for \Dyck{k}}

The idea was first to search for a basic adversary (\autoref{th:basicadv})
for small value of $k$. The efforts were focus on $k=2$. The goal was to find
two sets of word from $\{0,1\}^n$ that would have satisfied the requirements
of the basic adversary. Moreover, this two sets should give a lower bound of
the form $O(\sqrt{n}f(n))$ (for some function $f(n)$) in order to increase
the current lower bound of $O(\sqrt{n}c^2)$ (for some constant $c$).
But, this basic adversary returns lower bounds of the form
$\sqrt{\frac{mm'}{ll'}}$ where $m,m',l,$ and $l'$ (depending on $n$)
specify properties on both sets. I do not have found such sets,
more precisely, it is already difficult to think about a way
to define two sets such that their sizes increase nicely according
to a function in $n$. So, having to deal in the same time with
their associated values $m,m',l,$ and $l'$ make the task really difficult.


\subsection{An attempt to find a new reduction from easier problems}

In order to find a new lower bound, one idea was to try new reductions.
For $k=2$, some languages have been tested but an optimal algorithm
has been found before it worked. For $k=3$, one of the most interesting
language is defined by $c^*(ac^*bc^*)^*$.
In order to reject a word $w$, it is sufficient to search if a substring $v$
from $bwa$ can be part of the language $ac^*a+bc^*b$. So, this language is also
a star-free language (the same arguments than the one for \Dyck{k} would work
here). It means that its quantum query complexity is in $\tilde{\Theta}(\sqrt{n})$.
Moreover, it is possible to reduce this problem to \Dyck{3} simply by doing
the rewriting operations $a \mapsto 00, b\mapsto 11$ and $c \mapsto 01$. So,
finding the quantum query complexity of this problem can be useful in order
to get a new lower bound on the quantum query complexity of \Dyck{3}.